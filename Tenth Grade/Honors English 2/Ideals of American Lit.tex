\documentclass[12pt,twoside]{article}
\usepackage[top=1in,bottom=1in,left=1in,right=1in]{geometry}
\pagestyle{empty}
\linespread{2.0}
\fontfamily{phv}
\title{Ideals and Values of American Literature}
\author{Jesse Evers}
\date{9/5/14}
\begin{document}

\noindent Jesse Evers\newline
Ms. Hopkins\newline
English\newline
Period 2\newline
\centerline{\textbf{Ideals and Values of American Literature}}

\indent Unity, the state of being joined as a whole, is a cultural ideal of American literature. An ideal is something that is regarded as a perfect or near perfect quality in something or someone. Most nations strive for unity. Being united as a country creates a dynamic of neighborliness and of looking out for your fellow man. In \textit{I, Too, Sing America}, Hughes says ``I, too, am America,'' which implies that America is not, in fact, a physical area, but a community of people who are united as a cohesive group. This depicts a deep belief of American unity. Lynn Emanuel, in her peom \textit{Out of Metropolis}, says that she wants to feel ``ourselves holding the whole great story together.'' Emanuel thinks that the only reason America has gotten to the powerful place where it is is through American citizens uniting and working together to achieve amazing things. Had everyone remained isolated from each other, America would not be the world superpower that it is today. Whitman says ``I hear America singing,'' and goes on to describe the many different jobs that people are performing while singing. Whitman is saying that despite the obvious individuality of each singer, the fact that they are all singing together unifies them. He believes that even people who are extremely different from each other can come together to make a unified group. All three of the poets quoted include unity as a major theme in their poem. Ideals are an important thing to have in any community, whole countries included, because they provide a common goal for the entire population to strive for.

\end{document}